\setauthor{Marcel Pouget}

\section{Projektentstehung und Planung}
Die Idee entstand, nachdem sich eine Freundin darüber beklagt hatte, dass die Lern-App Quizlet eine kostenpflichtige Version auf den Markt gebracht hat. Dies brachte unser Projektteam dazu, das gerade auf der Suche nach einer neuen, umsetzbaren Idee war, sich mehr mit dem Thema zu beschäftigen. Schnell sind wir zu dem Entschluss gekommen, dass es zwar einfach zum Umsetzen wäre, jedoch keine große Neuerung dadurch entstehen würde. Da wir von der Schule die Anweisung bekommen haben, innovative Ideen für den Schutz der Umwelt umzusetzen, mussten wir ein bisschen umdenken. Dadurch entstand dann die endgültige Idee, eine Website, die sich auf Umweltfragen und Themen spezialisiert, zu entwickeln. Nach ersten Absprachen mit unseren Projektbetreuern und dem Abteilungsvorstand wurde uns das "OK" für die Umsetzung des Projekts gegeben. 

Das Ziel des Projektes ist es, bei der jungen Bevölkerung ein Bewusstsein für die Umwelt zu schaffen. Wir wollen den Lehrer*innen in den Stunden vor den Ferien die Möglichkeit geben, sich zuerst über spezielle Bereiche des Umweltschutzes zu informieren und dieses Wissen mit einem interaktiven Quiz zu festigen. Dadurch sollen vor allem bei jungen Menschen ein Umdenken ausgelöst werden. Sie sollen unsere Welt schätzen lernen. Wir wollen erreichen, dass es nicht nur darum geht, immer mehr zu haben. Unser langfristiges Ziel ist es, bei plus minus null herauszukommen. Eine Welt, in der wir mit der Natur im Einklang leben können, ohne dass wir immer das Neuste und Beste haben müssen. Und die ersten, kleinen Schritte zu dieser Zukunft, werden die Generationen, die jetzt auf die Welt und in die Schule kommen, einleiten. Und genau diese Menschen wollen wir erreichen, und einen fruchtbaren Gedanken “einpflanzen”. Ihnen zeigen, wie weit die Verschmutzung unserer Welt schon vorangegangen ist, und ihnen Lösungsansätze geben. Dadurch werden Generationen aufwachsen, die schon von Anfang an nachhaltig denken, und mit neuen und kreativen Ideen für die Rettung der Welt beitragen werden. 

   

Wir berufen uns bei der Idee des Projektes auf die These, dass man sich durch das Wiederholen von Informationen diese schneller merken kann. Dies möchten wir vor allem durch die Fragen in einem Quiz am Ende einer Lerneinheit ermöglichen, da dies die Kinder dazu bringt, sich mit dem zuvor Gelernten noch einmal auseinanderzusetzen. Unser Ziel ist es, ein nachhaltiges Bewusstsein für die Umwelt in den Köpfen der Schüler\*innen zu hinterlassen.  

Unsere Informationen beschränken sich bei der Planung nicht nur auf wissenschaftliche Artikel über das Lernen, sondern reichen von persönlichen Erfahrungen bis hin zu den Tipps und Ratschlägen unseren Professor\*innen. Vor allem die wissenschaftlichen Informationen der Autoren Alfred K. Treml \& Nicole Becker aus dem Buch "Einführung in Grundbegriffe und Grundfragen der Erziehungswissenschaft" waren für unsere Nachforschungen sehr interessant. Nachdem wir mit den für unser Projekt zuständigen Professor\*innen eine kleine Machbarkeitsstudie organisiert haben, ging es schon zur Planung der Umsetzung.  

Nach einigen Überlegungen haben wir uns beim Projektablauf für das agile Modell „Scrum“ entschieden, da es für unsere Zwecke die meisten Vorteile bietet. Der wichtigste Punkt ist dabei, dass unsere Professoren bei dieser Art der Projektentwicklung jede Woche den genauen Stand des Projektes kennen und so bei Schwierigkeiten den Fokus auf die Probleme legen können. Außerdem gibt es so jede Woche einen Check über die Qualität unserer Arbeit und wir können in besonderen Fällen im nächsten Sprint die Defizite der letzten sieben Tage nachholen. Ein anderer Punkt ist, dass dieses Vorgehensmodell bei uns in der Schule große Beliebtheit erlangt hat und auch vor allem für uns Schüler eine gute Möglichkeit ist, selbst etwas von dem Management größerer Projekte kennenzulernen. Die Meilensteine unseres Projektes haben wir einfachheitshalber auf größere Ziele begrenzt. Dabei war der erste Wichtige Punkt die Abschließung der Planung und der Projektstart. Der nächste Meilenstein war die erste Umsetzung eines Prototyps, welcher im Design schon dem endgültigen Produkt entspricht, jedoch noch keine Funktionen besitzt. Nachdem wir den ersten Clickdummy entwickelt haben, setzten wir uns daran, die Funktionen auf Localhost umzusetzen. Die letzten zwei Meilensteine, die wir noch vor uns haben sind das Deployment, und das Füllen der Website mit Informationen. 

\section{Inhaltliche Beschreibung der Projektumsetzung und Ausblick}

Das Hauptaugenmerk unseres Projektes ist die Beschaffung von Informationen, die Aufbereitung dieser und das anschließende Quiz, um das gelernte spielerisch wiederholen und so lernen zu können. Wie schon beschrieben setzen wir hier vor allem auf Lehrer, welche ihren Schülern in einer Unterrichtseinheit eine sogenannte „Lesson“ vortragen. Diese besteht immer aus einer Sektion, einem Titel, Bildern und Grafiken, und natürlich schriftlichen Informationen und Fakten. Der Vortrag einer Lesson soll ca. 20 Minuten dauern, da die Aufmerksamkeitsspanne der Schüler bei rund 10-15 Minuten liegt. Um diese nicht zu überstrapazieren, werden die Informationen einfach verständlich und visuell ansprechend aufbereitet. Um die Aufmerksamkeit der Kinder wieder für uns zu gewinnen, bieten wir pro Lesson ein Quiz an, an welchem die ganze Klasse teilnehmen kann. Dafür muss man einen QR-Code scannen oder eine Game ID auf der dafür vorgesehenen Unterseite eingeben. Der Lehrer präsentiert nun die Frage mit den vier Antwortmöglichkeiten auf der Wand oder einem Smartboard, und die Schüler können dann für die richtige Antwort abstimmen. Am Ende sieht man eine Statistik über die ganze Klasse und bekommt Vorschläge, wie man das Wissen der Schüler bei den Punkten, wo sie schwächer abgeschnitten haben, verbessern kann. Auch von der technischen Seite der Umsetzung setzten wir auf ganz neue Methoden. So wird das ganze Projekt vorerst in der schuleigenen „Leocloud“ gehostet. Dort laufen dann die ganzen Applikationen der App in einem Kubernetes Cluster, um je nach Bedarf genug Ressourcen für unsere User bereitstellen können. Das besondere an Kubernetes ist, dass falls ein sogenannter Pod an seine Leistungsgrenzen stößt, einfach ein neuer Service in einem neuen Pod gestartet werden kann. Dies geschieht automatisch, und ist dafür da, um bei größeren Mengen an Anfragen automatisch die Leistung der Software zu erhöhen, und somit allen Nutzern immer die beste Performance zu bieten. 

Das Besondere an unserem Projekt ist das Verknüpfen von aufbereitetem Wissen und des spielerischen Lernens. Da wir uns auch für das Sammeln der Informationen kümmern, haben wir als Administratoren immer einen Überblick über die Verlässlichkeit der Inhalte, können diese anpassen und erweitern. Im Allgemeinen ist uns bei unserer Recherche aufgefallen, dass es zwar vereinzelte Blogs im Internet gibt, die vereinzelt ein Quiz anbieten, jedoch findet man keine Seiten im Internet, wo es sowohl gut aufbereitete Informationen und Fragen dazu gibt.  

Um das ganze Vorhaben wahr werden zu lassen, nutzen wir den Unterricht, den wir die letzten Jahre an der HTBLA Leonding genießen durften. Um die Fragen und Informationen für jeden problemlos zugänglich zu machen, haben wir uns gegen eine klassische App und für eine Web-Applikation entschieden. Dies hat den großen Vorteil, dass man sowohl vom Smartphone als auch von jedem PC oder Laptop auf unsere Seite zugreifen kann. Wir haben uns bei der Umsetzung des Clients für das Framework Angular in der Version 14 entschieden, da uns dieses von unseren Professoren beigebracht wurde, und es für unsere Zwecke genau die richtigen Funktionen unterstützt. Unser ganzes Team hat schon viel Erfahrung mit der Entwicklung von Angular, und aus diesem Grund haben wir uns für diese Möglichkeit entschieden. Bei der Umsetzung des Servers war die Entscheidung schon etwas schwerer, da wir zwischen Node.JS und Java mit dem Framework Quarkus geschwankt sind. Letztendlich haben wir uns für Java und gegen Node.js entschieden, da wir mehr Erfahrung damit haben. Die Funktionen werden mit Hilfe von Web-Sockets umgesetzt, wobei uns Quarkus sehr unter die Arme greift. Das Java-Framework Quarkus ist eine neue, schnelle Erweiterung, die es mit wenig Ressourcen-Aufwand erlaubt, sogenannte Rest Endpunkte in Java zu erstellen. Auch die Anbindung an eine Datenbank wird unterstützt, und kann mit der Hilfe des oben genannten System Kubernetes perfekt genutzt werden. Für uns erfüllt es genau unsere Anforderungen, da wir auf der Suche nach einer performanten, aber praktischen Lösung waren.  

  

  

Wir nutzten Quarkus auch für das Abspeichern der Daten, der Verbindung zu einer PostgreSQL Datenbank, und für andere Aufgaben des Backends. Bei der Umsetzung des Projektes achten wir stets darauf, dass wir alle Sicherheitslücken gleich von Anfang an schließen, und auch sonst jegliche Standards einhalten. Wir speichern keine Userdaten ab, und bieten für die Schüler keine Möglichkeit, sich zu registrieren. Die Statistiken der Schüler werden nur auf Wunsch des Lehrers in die Datenbank persistiert, und es werden keine persönlichen Daten gespeichert oder verkauft.  

  

Das Projekt steht mitten in der Entwicklungsphase, auch wenn diese sich langsam dem Ende nähert. Die Features, welche von uns am Anfang geplant wurden, sind schon weit in der Entwicklungsphase, und stehen kurz vor dem Abschluss. Der nächste große Schritt wäre es, die Datenbank mit Informationen und Daten zu füttern, um die Website auch wirklich benutzbar zu machen. Bis jetzt steht die Website, mit der Möglichkeit, Inhalt den Schülern zu präsentieren, und danach ein Quiz zu starten. Außerdem haben wir ein kleines Teaser-Video produziert, um bei uns in der Schule schon auf unser Projekt aufmerksam zu machen. Den Link zu dem Video finden Sie am Ende des Protokolls. Eine (fast) fertige Version der Seite ist seit dem 24.01. auf der Leocloud aufrufbar, jedoch gibt es mit dem Hauptfeature, dem Quiz noch ein Problem in der Cloud selber, da aus Sicherheitsgründen noch keine Websocket-Verbindung auf einen Kubernetes-Pot gemacht werden kann. Wir stehen schon mit den Professoren in Verbindung, um dieses Problem schnellstmöglich zu beheben, und erste User auf unsere Website zu locken. Die erste Version der Website können Sie hier aufrufen. Die fertige Spielbarkeit eines Quiz ist bis 6.2 geplant, wir müssen bis dahin aber noch uns mit den zuständigen Professoren austauschen, damit auch dieses Problem reibungslos hinter uns gebracht werden kann.  

 

Die Qualität der bisherigen Ziele wurde wöchentlich von unseren geschätzten Professoren kontrolliert. Falls ein Ziel nicht zufriedenstellend abgeschlossen wurde, haben wir uns die Woche danach hingesetzt, um das erwartete Ergebnis zu liefern. Probleme dabei gab es hauptsächlich während der Phasen, in denen auch andere Professoren von uns Projekte, Hausaufgaben und Tests verlangt haben, da wir in dieser Zeit etwas länger gebraucht haben, um unsere Features umzusetzen. Weitere Probleme gab es bei der Veröffentlichung der Website auf der Leocloud, da diese, wie oben beschrieben, aus Sicherheitsgründen noch keine Websockets zulässt, jedoch setzten wir alles daran, das Problem zu lösen und das Quiz spielbar zu machen. 

  

  

Leogreen ist für die Schüler der HTL Leonding in der hauseigenen “Leocloud” verfügbar, und am Ende der Entwicklung auch spielbereit. Weitere Schritte der Veröffentlichung hängen dann von dem Erfolg der Website im Umfeld der HTL ab. Es wird auch eine Version des Projektes auf einem kleinen Server von Oracle für die Öffentlichkeit zur Verfügung stehen, dieser dient aber eher dem Zweck, das Projekt bei Bedarf Interessenten zu präsentieren. Sollte unsere Website gut ankommen und von vielen Menschen innerhalb der Schule genutzt werden, werden wir natürlich alles daransetzen, die Website auf größeren Servern aufzusetzen, um eine größere Nutzeranzahl zu ermöglichen. (Auf der Leocloud wird das Projekt leider nur bis zu unserem Schulabschluss gehostet bleiben, da immer die Namespaces der alten Schüler gelöscht werden, um Speicherplatz für neue Schüler bereit zu stellen) 

Durch kleine Teaser-Videos und von uns erstellten Werbefilmen könnten wir dann unsere Seite in der breiten Öffentlichkeit etablieren. Dies hätte zur Auswirkung, dass sich mehr Menschen für unsere Themen interessieren und sich mehr mit der Umwelt beschäftigen. Auch bei anderen Schulen möchten wir später Fuß fassen, um vor allem der jungen Generation dort ein Gefühl für unsere Mutter Erde zu geben.  

Unsere Zielgruppe nach Abschluss des Projektes sind die Schüler zwischen 13 und 20 Jahren. Denn dies ist die Zeit, in der wir uns die größte Aufmerksamkeit erhoffen. Außerdem möchten wir damit erreichen, das Bewusstsein nachhaltig zu verändern. Um dies zu testen, werden wir, wie oben beschrieben, nach Abschluss eine Testphase an der HTL Leonding einleiten. Unser Ziel ist es, das Verhalten der Schüler zu beobachten.  Durch die Beobachtung können wir daraus schließen, ob unser Projekt Früchte tragen wird, oder ob sich zu wenige Menschen für dieses Thema interessieren. Um unsere Zielgruppe das erste Mal auf uns aufmerksam zu machen, wird es im März einen sogenannten „Project Award“ an unserer Schule geben, wo unser und einige andere Projekte (auch von anderen Schulen) einem größeren Publikum vorgestellt, und mit Preisen angepriesen werden.  

Kooperationen mit anderen Schulen als der HTBLA Leonding gibt es leider nicht, jedoch werden wir hiervon allen Lehrern rückhaltlos unterstützt, und bekommen alle Ressourcen zur Verfügung, die wir für die Umsetzung der Idee brauchen. Sowohl Zeit, die technischen Mittel als auch Hilfestellungen werden uns von den Professoren zur Verfügung gestellt, und erlauben es uns, reibungslos zu arbeiten.  

\subsection{Ausblick}

Das Projekt Leogreen hat sehr viel Potential, auch nach Abschluss noch regelmäßig erweitert zu werden. So gibt es jetzt schon Pläne, verschiedene Spiel-Modi zu entwickeln, oder ein eigenes Content-Management-System (cms) umzusetzen, um den User die Möglichkeit zu geben, jederzeit neue Inhalte hinzuzufügen, ein neues Quiz zu erstellen, oder eigene Lerneinheiten zu organisieren. Um dies ermöglichen zu können, benötigt es aber von unserer Seite aus noch einen Haufen an Zeit, und auch finanzielle Mittel, um nach Abschluss der Schule die Ressourcen anzuschaffen, um das Projekt am laufen zu halten. Natürlich würde es dauerhafte Updates geben, die kleine Bugs oder andere Fehler ausbessern würden, und auch die Benutzbarkeit wird nach Abschluss noch verbessert werden.  

Es ist schwer zu sagen, was in Zukunft mit dem Projekt passieren wird. Am besten wäre es, wenn es bei den Schülern und in der Öffentlichkeit so eine große Aufmerksamkeit erreichen würde, dass es sich selbst finanzieren könnte. Das teuerste bei der Instanthaltung wären nämlich etliche Kosten für Server, die Domain oder andere Wartungskosten. Sollte es sich wirtschaftlich nicht rentieren, also zu wenige Nutzer die Website besuchen, wird das Projekt auf einem kleinen Server von Oracle aufrufbar sein, jedoch nur in unserer Freizeit weiterentwickelt werden. Es würde also kaum bis sehr wenige Updates geben, und auch die Performance der Website würde zu wünschen übriglassen. 

 
 

Mit den gewonnenen Erkenntnissen aus unserem Projekt könnten in der Zukunft noch weitere Projekte geplant werden. Konkret geht es darum, wie man am besten bei einer vorher festgelegten Zielgruppe nachhaltig Wissen näherbringen kann. Da dieser Anwendungszweck sich nicht nur auf Umwelt-Themen einschränken lässt, wäre es durchaus einer Überlegung wert, einige Bereiche der Schule nur noch durch diese neue Art des Lernens zu ersetzten. So könnte man zum Beispiel Fächer wie Geschichte oder Geographie so gestalten, dass es am Anfang einer Stunde einen Input vom Lehrer gibt, der nicht länger als 20 Minuten dauert. Falls es danach noch Fragen geben sollte, kann die zuständige Lehrkraft diese noch vor dem Quiz beantworten. Das Quiz dient dann zur Festigung des vorher übermittelten Wissens, und danach kann der Lehrer die Statistiken auch dazu nutzen, um Schüler, welche nicht so gut abschneiden, zu fördern, und zu helfen, dass diese sich dem Durchschnitt der Klasse anpassen. Außerdem könnte man diejenigen, welche überdurchschnittlich gut abschneiden, fördern, um somit die Talente und Schwächen der einzelnen Individuen erkennen. Man würde dann nicht mehr die Klasse als Ganzes, sondern den Schüler als einzelne Personen vor sich haben. Man könnte ohne Schwierigkeiten das System mit anderen Lehrern synchronisieren, und somit einen Austausch zwischen den Fächern und Lehrpersonen ermöglichen. Natürlich darf dabei aber nie die Menschlichkeit einer Person aus den Augen gelassen werden. Es sollte also niemals nur nach den gespeicherten Daten, sondern auch durch Mitarbeit und dem persönlichen Eindruck des Lehrers benotet werden.   

 

Das Konzept, spielerisch zu lernen hat natürlich auch in anderen Bereichen großes Potential. Wenn es um Führerscheinfragen geht, wird es ja schon so gemacht, dass man durch multiple Choice-Fragen die Regeln des Straßenverkehrs eingeprägt bekommt. Aber wieso nicht ein bisschen abwechslungsreicher gestalten, mit einem Text und dann Fragen dazu? Oder auch bei einer Lehre, beim Kochen oder sonstigen Beschäftigungen, wo man sich was beibringen möchte. Überall könnte man das Konzept von Leogreen, von unserem Projekt, anwenden.  

Wir müssten zwar vor allem am Anfang die Menschen zum Umdenken bewegen, aber genau dafür sind wir da. Um die Menschen zum Umdenken bewegen. Und genau dafür haben wir das Projekt gestartet. Um wichtige Daten zu bekommen, und um zu überprüfen, ob durch unsere Website wirklich Wissen bei den Schülern hängen bleibt. Dafür werden wir selbst nach dem Abschluss des Projektes noch eng mit den Professoren an unserer Schule in Kontakt stehen, um so die neusten Informationen über den Wissensstand über die Schüler zu bekommen. Dies geschieht mit den vorgegebenen Regeln und Richtlinien, sodass keine persönlichen Daten der Schüler zu uns oder an die Öffentlichkeit geraten. Man könnte also in sehr weiter Zukunft ein regelrecht neues System an Schulen entwickeln, die auf ganz neue Art und Weise Inhalte weitergibt. Ganz zu schweigen von dem Impakt, der unser Projekt auf die Gesellschaft haben könnte, wenn sich die Website weit genug verbreiten würde. Es würden viele, junge Menschen aktiv mit unserer Umwelt auseinandersetzten, neue Wege finden, diese zu schützen, und im bestenfalls auch ihre Eltern dazu überreden, mal nicht mit dem Auto zu fahren, sondern Öffis zu nehmen. Mit genug Reichweite und Nutzerzahlen, könnten die Erkenntnisse des Projektes eine wirklich große Veränderung in unserer Denkweise hervorrufen. Und auch wenn Leogreen nicht die gewünschte Aufmerksamkeit bekommt, ist es für unser Team eine wirklich gute Übung, um in Zukunft noch weitere, bessere Projekte umzusetzen. Im Besten fall verändern wir die Welt, im schlimmsten Fall haben wir was gelernt. So oder so, die gewonnen Erkenntnisse, egal ob sie für jeden oder nur uns weitergebracht haben, sind das Wertvollste, was wir aus diesem Projekt mitnehmen können. 

Denn wir möchten vor allem am Anfang auf die Ressourcen der Schule zurückgreifen, um das Projekt kostenlos für alle zur Verfügung zu stellen. Auch im späteren Verlauf sollten die selbsterhaltungskosten nie überschritten werden, da wir uns als Non-Profit Organisation sehen. Die Ausgaben, welche wir für die Wartung und die Server tätigen müssen, möchten wir durch Werbe-Flächen decken. Kostenpflichtige Accounts oder sonstige Dienstleistungen sind nicht geplant. Selbstverständlich muss die Website eine gewisse Nutzeranzahl bekommen, da sonst auf lange Sicht nicht genügend Einnahmen durch die Advertisements getätigt werden können, um Strom und Internetkosten zu decken. Auch die oben erwähnten Server, Hardware und Entwicklungskosten dürfen nicht überschätzt werden. Wir als Team arbeiten jedoch nicht Profit orientiert, uns ist es das Wichtigste, dass sich das Projekt finanziell selbst erhalten kann, selbst wenn wir nicht mehr auf die Vorteile der Schule zurückgreifen können. 

\section{Bericht der Projektkoordinatorin bzw. des Projektkoordinators}

Die Zusammenarbeit innerhalb des Projektteams lief fast reibungslos ab. Einzig die Kommunikation hat vor allem durch andere Fächer gelitten. So musste jeder von uns an anderen Tagen etwas erledigen, und es gab oft Situationen, wo es nicht jedem klar war, welche Aufgaben er im neuen Sprint erledigen musste. So war am Anfang Dominik vor allem für das Design zuständig, während Lorenz und Fabian sich schon an die Umsetzung der Website in Angular gemacht haben. Marcel hat den Teil des Teamleiters übernommen und sich an die Ideen und den Funktionen der Website gesetzt. Sein Teil war es, sich zu überlegen, wie die Unterrichtseinheiten aufgeteilt werden und wie ein Quiz funktionieren soll. Im weiteren Verlauf des Projektes haben sich unsere Aufgabenbereiche verändert. So ist Dominik nun für den Server und das Deployment verantwortlich, während  

Fabian und Lorenz sich weiter um die Umsetzung der Features kümmern. Marcel war für den Rohschnitt des Videos und kleinere Anpassungen an der Website und organisatorische Arbeiten zuständig.  

   

Die Zusammenarbeit mit den Professoren lief bis auf ein paar Missverständnisse auch gut ab und wird hoffentlich auch in der Zukunft noch so bleiben. Vor jedem Sprint, also jeden Montag, haben wir uns zusammengesetzt und gemeinsam mit ihnen den Fortschritt besprochen und für die neue Woche Ziele festgelegt. So konnten wir, immer wenn eine Woche mal etwas stressiger war, und es andere Aufgaben zum Erledigen waren, wurden die Ressourcen so aufteilen, dass weder das Projekt noch andere Fächer zu kurz kamen. Geleitet haben das Projekt Aberger Christian und Hammer Hans-Christian, welche uns in dem Fach "Informationstechnische Projekte" unterrichten.  

   

Unter den Teilnehmern war der Aufwand des Projektes immer gut verteilt, jedoch hatten Lorenz und Fabian mit der Umsetzung der Features oft mehr Arbeit als Marcel und Dominik, da diese sich besser in den für die Umsetzung verwendeten Sprachen auskennen und mehr Erfahrung darin haben. Dadurch empfanden Fabian und Lorenz die Arbeit auch als deutlich fordernder.  

   

Für die Kommunikation untereinander benutzen wir das kostenlose Chat-Programm "Discord". Über diese Kanäle werden Aufgaben verteilt, Fortschritte besprochen und Ideen geteilt. Mit den Professoren wird in der Schule jeden Montag der Fortschritt geprüft und neue Ziele mündlich festgelegt. Um diese immer im Blick zu haben, werden diese auch auf Github unter dem Tab "Projects" gespeichert. Hier werden auch, falls jemand an einem Ziel arbeitet, die verschiedenen Statusanzeigen live aktualisiert. So kann jeder immer sehen, wie weit die Ziele sind und was noch zu machen ist. Für die Professoren ist dies ein guter Weg, um zu überprüfen, wer wann was gemacht hat. Falls wir Fragen haben, dürfen wir auch jederzeit zu den Professoren gehen und diese stellen.  

   

Erst durch die Kenntnisse, die wir in der Schule bekommen haben, war es für uns möglich, dieses Projekt zu starten und so weit zu bringen. Viele der von uns benutzten Programme, Programmiersprachen und Librarys, die uns von Professoren*innen unserer Schule nähergebracht wurden. 

Es stecken 250 Stunden Arbeitszeit in dem Projekt. Größere Probleme oder  

Rückschläge gab es bisher nicht, einzig wird die Zeit bis zum geplanten Abschluss sehr knapp, da wir ab Mai offiziell nicht mehr Schüler an unserer Schule sind. Probleme mit der Einhaltung von Terminen hatten wir vor allem in den Phasen, wo viele Fächer Test, Schularbeiten oder Projekte von uns wollten. Ansonsten war es für uns immer möglich, die Fristen einzuhalten. 

Verträge mit Partnern gibt es nicht, der einzige Vertrag, den wir abgeschlossen haben, ist der Projektvertrag mit unseren Professoren.  

  